\vspace*{\fill}
\noindent

% \textit{We believe that students can exert their creativity without being constrained by the information overload they are faced with throughout the educational process.} 

% \textit{The university is a complex environment which sometimes tends to develop chaotically, an evolution that is visible in the vast array of software applications that exist to assist students. Students are expected to be aware of everything that is posted on these platforms, to discern between relevant information and spam, while also busy attending courses and completing assignments.}

% \textit{Through the power of the community, students could more easily flag and organize important information from these websites. This paper describes a collaborative app that allows students to better organize their time and schedule by collectively identifying the most useful information from the university's sources and intelligently sharing it with their peers.}

% \textit{We believe that students can exert an abundant amount of creativity without being constraint by the information overload that exists in universities. This creativity can sometimes develop chaotically if the course platforms, wikis created by professors or departments are not student assisting first. }

% \textit{Through the power of the community to flag and organize important information from these websites, disseminate it to relevant colleagues, this paper describes a collaborative app that allows students to better organize their time and schedule by collectively identifying the most useful information from the university's sources and intelligently sharing it with their peers. }

\textit{The university is a complex environment with many creative people having as a disadvantage its sometimes chaotic development. This evolution is visible in the vast array of software applications meant to assist students. Over time, in our university, many such platforms were developed, including the university website, the faculty website, websites of different student organizations, wikis created by professors or departments, and the official course platform. This variety turned into an information overload for students who are expected to be aware of everything that is posted on these websites, to discern between relevant information and spam, while also being busy attending courses and completing assignments.}

\textit{We believe that students themselves can address this issue by using the power of their community to flag and organize important information from these websites, disseminate it to relevant colleagues and help each other get through the labyrinth of information in the university.}

\textit{With this goal in mind, this paper describes a collaborative app that allows students to better organize their time and schedule by collectively identifying the most useful information from the university's sources and intelligently sharing it with their peers.}

\vspace*{\fill}