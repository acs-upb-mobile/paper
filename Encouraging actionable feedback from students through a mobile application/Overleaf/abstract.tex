\vspace*{\fill}
\noindent

\textit{Acțiunea de a împărtăși feedback se regăsește în caracterul fiecărui om. În absența unor opinii constructive din partea celorlalți, probabilitatea ca oamenii să își dezvolte abilitățile și să evolueze în carieră scade dramatic.}

\textit{Universitățile oferă adesea oportunități studenților de a-și împărtăși impresiile pe diferite tematici. Cu toate acestea, studenții sunt deseori reticenți sau prea puțin încurajați. În plus, recenziile împărtășite între diferite generații de studenți sunt complet dezorganizate. Prin urmare, această teză explorează motivele care determină, de obicei, studenții să fie dezinteresați de acțiunea de a oferi feedback și se scufundă în înțelegerea nivelului lor de apreciere cu privire la utilitatea acestuia. Mai mult, contribuim la scăderea haosului general, generat de toate opiniile distribuite prin intermediul platformelor de socializare sau chestionarelor neoficiale aleatorii, prin crearea unui mediu cuprinzător.}

\textit{Analizăm situația existentă a feedback-ului în mediul educațional și investigăm nevoile studenților. Pe baza ideilor colectate, propunem o soluție reprezentată de o aplicație mobilă ce utilizează sisteme și instrumente pregătite de utilizat în producție pentru a oferi stabilitate și scalabilitate. Aceasta conține diverse module concepute pentru a spori conștientizarea asupra a ceea ce înseamnă feedback constructiv și oferă ușurință în utilizare, transparență și un set de date structurat.}

~
\vspace*{2\baselineskip}

\textit{The act of sharing feedback lies in the character of each human. In the absence of constructive opinions from others, the likelihood of people developing their abilities and evolving their careers decreases dramatically.}

\textit{Universities often offer opportunities to their students to share their impressions on different topics. However, students are often reluctant or too little encouraged. In addition, the reviews passed between different generations of students are entirely disorganized. Hence, this thesis explores the reasons that usually prompt students to be disinterested in the action of providing feedback and dives into understanding their level of appreciation on the usefulness of it. Moreover, we contribute to decreasing the overall chaos generated by all the opinions distributed via social media platforms or randomly unofficial questionnaires by creating a comprehensive environment.}

\textit{We analyze the existing situation of feedback in the educational environment and investigate the needs of students. Based on the ideas collected, we propose a mobile application solution that uses production-ready systems and tools for offering stability and scalability. It contains various modules designed to broaden the awareness of what constructive feedback means and provide ease of use, transparency, and a structured data set.}

\vspace*{\fill}