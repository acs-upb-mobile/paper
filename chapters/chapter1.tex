\chapter{Introduction} \label{chapter1}

\section{Motivation} \label{1:motivation}

    According to Kevin Kelly \cite{kelly2010technology}, \textit{technology is an extension of life.} Its roots in our day to day lives go so deep that, for most of us, it is impossible or extremely difficult to even imagine life without technology. It has slithered its way into our jobs \cite{lewis1996studying}, our commute \cite{kairi2019technology}, even our nutrition \cite{lewis2010role} and personal lives \cite{mcquillen2003influence}. It should come as no surprise that it has become a huge part of students' lives, which is what we will be focusing on hereafter.
    
    ~
    
    Due to the rapid pace in which new and better technologies replace the old ones, as well as the existence of a variety of tools that serve more or less the same purpose, students nowadays find themselves flooded with a plethora of platforms and resources provided by their faculty.
    
    The emergence of new and better tools that replace old ones often implies that some people will continue to use the old tools (or even "analogue" tools instead of "digital" ones) out of habit, while other, more progressive people will immediately switch to the new ones. This leads to a discrepancy in the platforms used by professors for the same purpose, which in turn overwhelms students to the point where some even drop out.
    % TODO: Link to case study with used technlogies in ACS
    
    In order to cope with the huge influx of information (new knowledge, as well as administrative details), students oftentimes turn to each other for help and assistance. This process is informal in nature, taking place either face to face or through social media platforms.
    % TODO: Link to case study with socialization in ACS
    \clearpage
    This paper proposes a cross-platform mobile application which aims to act as a compendium for the various resources offered by the university for the students, as well as provide an easy-to-use, intuitive platform for collaboration.

\section{Environment} \label{1:environment}

    For the purpose of this paper and the application prototype, we will be analyzing the case of one of the 15 faculties\footnote{https://upb.ro/en/faculties/} of \textbf{\acrlong{upb}} in Romania, namely the \textbf{\acrlong{acs}}. From here on out, we will be referring to the larger engineering institution as \textit{the university} or \textit{\acrshort{upb}}, and to the computer science faculty as \textit{the faculty} or \textit{\acrshort{acs}}.
    
    \subsection{Student/user profile} \label{2:student_profile}
    
        The student using this app has the following characteristics:
        \begin{itemize}
            \item an ongoing \textbf{Bachelor's/Master's degree} in computer science
            \item average to high \textbf{digital proficiency}, common for computer science students%, therefore usage instructions should be minimal.
            \item proficiency in either \textbf{Romanian or English}, the two languages that the institution provides courses in%, therefore the application
            \item an age between \textbf{18 and 25 years}
        \end{itemize}
        
    \subsection{Student body structure} \label{1:student_body_structure}
    
        The faculty provides Bachelor's, Master's and Doctoral science degrees. We are interested in the B.Sc. and M.Sc. degrees, each of which are split into a couple of \textit{domains}. They last for four and two years respectively.
        
        For the \textbf{B.Sc.}, each domain has its own separate curriculum, and the students are organised hierarchically into \textit{series}, which are comprised of up to 5 \textit{groups}, which in turn can be split into two \textit{subgroups} each.
        
        For the \textbf{M.Sc.}, each domain hosts a number of master's programs, and each program corresponds to a single \textit{series} of students which is not split further.
        
        Each \textit{series} and each \textit{group} have a student representative who acts as a link between the corresponding group of students and the faculty, and is tasked with things like passing on important information, keeping in touch with the Student Office staff and making sure the contracts are signed by all students each semester.
    
    \subsection{University platforms} \label{1:university_platforms}
    
        The main platforms that the University staff use to provide resources to the students are:
        
        \begin{itemize}
            \item \textbf{the official university website} for general university information such as the campus map and academic calendar
            \item \textbf{the official faculty website} for the curriculum, certain announcements and timetable information
            \item \textbf{various course/wiki websites} for class resources
            \item \textbf{a \gls{moodle} instance} for class-related announcements, discussions and assignments
            \item \textbf{the university's administrative website} for the student contract, personal information and grades
            \item \textbf{several automated-testing environments} for coding assignments
        \end{itemize}
    
        Most of the aforementioned platforms can be accessed using a single set of credentials provided by the University upon admission.
        
    \subsection{Class structure} \label{1:class_structure}
    
    A student's schedule is comprised of three main types of classes:
    
    \begin{itemize}
        \item \textbf{lectures}: theoretical classes taught by professors or lecturers, which take part in a lecture hall and in which a whole series (or two, in some cases) of students participates
        \item \textbf{laboratories}: practical classes taught by professors, lecturers (rarely) or \acrshort{ta}s (\acrlong{ta}s), which require a computer or other external tools (e.g. lab equipment, electrical equipment etc.), which take part in dedicated lab rooms and in which only a group or subgroup of students participates
        \item \textbf{seminars}: practical classes taught by professors or lecturers, which only require pen, paper and a whiteboard, which take part in classrooms and in which only a group or subgroup of students participates
    \end{itemize}
    
    The main forms of evaluation are: theoretical exams, homeworks, projects, tests, practical exams and research assignments.
        
\section{Proposed functionalities} \label{1:functionalities}

    Given the environment described in section \ref{1:environment}, we can propose a number of main functionalities for the application:
    
    \begin{itemize}
        \item \textbf{authentication} (ideally linked to the common credentials used for the other platforms)
        \item \textbf{localization}: the application should provide a \acrshort{ui} (\acrlong{ui}) in both Romanian (for local students) and English (for international students)
        \item the possibility to \textbf{access, modify and filter a list of platforms provided by the university} (see section \ref{1:university_platforms}), as well as view information about their use
        \item the possibility to collaboratively \textbf{create and edit a timetable} which includes class information as well as various events
        \item \textbf{campus map and navigation information}
        \item \textbf{news, FAQs and other relevant information} that students can contribute to
    \end{itemize}

\section{Goals} \label{1:goals}

    The application aims to make students' lives easier by providing a single point of access for all university resources and needs, supported by the students, for the students. It should be intuitive and easy to use, as well as allow customization for any kind of student.
    
    Above all, the application should be particularly helpful for freshman students by offering an easy way to find out anything they want to know about the ins and outs of their new university life.
    
    Ultimately, the application aims to become an integral part of students' lives and contribute to their education, all the while decreasing the feelings of anxiety caused by being bombarded with new, often disorganised information.

\section{Outline} \label{1:outline}
    TODO