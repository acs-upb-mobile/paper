\chapter{Introduction} \label{chapter1}

\section{Motivation} \label{1:motivation}

    According to Kevin Kelly \cite{kelly2010technology}, \textit{technology is an extension of life}. Its roots in our day to day lives go so deep that, for most of us, it is impossible or complicated to even imagine life without technology. It has slithered its way into our jobs \cite{lewis1996studying}, our commute \cite{kairi2019technology}, even our nutrition \cite{lewis2010role} and personal lives \cite{mcquillen2003influence}. It should come as no surprise that it has become a huge part of students' lives, which we will be focusing on hereafter.
    
    ~
    
    Due to the rapid pace in which new and better technologies replace the old ones, as well as the existence of a variety of tools that serve more or less the same purpose, students nowadays find themselves flooded with a plethora of platforms and resources provided by their faculty.
    
    Nowadays, new and better technologies replace old ones at a rapid pace. This progress often implies that some people will continue to use the old tools (or even "analog" tools instead of "digital" ones) out of habit, while early adopters will immediately switch to the new ones. As a result, there exists a discrepancy in the platforms used by professors for the same purpose. This discrepancy, in turn, overwhelms students, particularly in their first year of university.
    % TODO: Link to case study with used technologies in ACS
    
    % TODO: mi se pare ca intri ff repede in studenti ca sa zic asa. 
    
    In order to cope with the massive influx of information, students often turn to each other for help and assistance. This process is informal, taking place either face to face or through social media platforms.
    % TODO: Link to case study with socialization in ACS
    \clearpage
    This paper proposes a cross-platform mobile application that aims to act as a compendium for the university's various resources, as well as provide an easy-to-use, intuitive platform for collaboration.

\section{Environment} \label{1:environment}

    For the purpose of this paper and the application prototype, we will be analyzing the case of one of the 15 faculties\footnote{https://upb.ro/en/faculties/} of \textbf{\acrlong{upb}} in Romania, namely the \textbf{\acrlong{acs}}. From here on out, we will be referring to the larger engineering institution as \textit{the university} or \textit{\acrshort{upb}}, and to the computer science faculty as \textit{the faculty} or \textit{\acrshort{acs}}.
    
    This section provides insight into the context of this application, namely the typology of the \acrshort{acs} student (subsection \ref{2:student_profile}) - our user - as well as relevant details about the way their university life is organised. The latter includes information about how the student groups are organised within the university (subsection \ref{1:student_body_structure}), which platforms they need to use in order to perform their duties (subsection \ref{1:university_platforms}) as well as what kind of activities make up their schedule (subsection\ref{1:class_structure}).
    
    \subsection{Student/user profile} \label{2:student_profile}
    
        The student using this app has the following characteristics:
        \begin{itemize}
            \setlength{\topsep}{0.5pt}
            \setlength{\itemsep}{0.5pt}
            \setlength{\parsep}{0.5pt}
            \item an ongoing \textbf{Bachelor's/Master's degree} in computer science
            \item average to high \textbf{digital proficiency}, common for computer science students%, therefore usage instructions should be minimal.
            \item proficiency in either \textbf{Romanian or English}, the two languages that the institution provides courses in%, therefore the application
            \item an age between \textbf{18 and 25 years}
        \end{itemize}
        
    \subsection{Student body structure} \label{1:student_body_structure}
    
        The faculty provides Bachelor's, Master's, and Doctoral science degrees. We are interested in the B.Sc. and M.Sc. degrees, each of which is split into a couple of \textit{domains}. They last for four and two years, respectively.
        
        For the \textbf{B.Sc.}, each domain has its separate curriculum, and the students are organized hierarchically into \textit{series}, which are comprised of up to 5 \textit{groups}, which in turn can be split into two \textit{subgroups} each.
        
        For the \textbf{M.Sc.}, each domain hosts several Master's programs, and each program corresponds to a single \textit{series} of students, which is not split further.
        
        Each \textit{series} and each \textit{group} have a student representative. This person acts as a link between the corresponding group of students and the faculty and is tasked with things like passing on important information, keeping in touch with the Student Office staff, and making sure the contracts are signed by all students each semester.
    
    \subsection{University platforms} \label{1:university_platforms}
    
        The main platforms that the University staff use to provide resources to the students are:
        
        \begin{itemize}
            \setlength{\topsep}{0.5pt}
            \setlength{\itemsep}{0.5pt}
            \setlength{\parsep}{0.5pt}
            \item \textbf{the official university website}\footnote{http://upb.ro/} for general university information such as the campus map and academic calendar
            \item \textbf{the official faculty website}\footnote{http://acs.pub.ro/} for the curriculum, certain announcements and timetable information
            \item \textbf{various course/wiki websites}\footnote{https://ocw.cs.pub.ro/, http://elf.cs.pub.ro/} for class resources
            \item \textbf{a \gls{moodle} instance}\footnote{https://acs.curs.pub.ro/} for class-related announcements, discussions and assignments
            \item \textbf{the university's administrative website}\footnote{https://studenti.pub.ro/} for the student contract, personal information and grades
            \item \textbf{several automated-testing environments}\footnote{https://vmchecker.cs.pub.ro/, https://v2.vmchecker.cs.pub.ro/} for coding assignments
        \end{itemize}
    
        Most of the platforms mentioned above can be accessed using a single set of credentials provided by the university upon admission.
        
    \subsection{Class structure} \label{1:class_structure}
    % TODO: cum se leaga asta de restul?
    
    A student's schedule contains three main types of classes:
    
    \begin{itemize}
        \setlength{\topsep}{0.5pt}
        \setlength{\itemsep}{0.5pt}
        \setlength{\parsep}{0.5pt}
        \item \textbf{lectures}: theoretical classes taught by professors or lecturers, which take part in a lecture hall and in which a whole series (or two, in some cases) of students participates
        \item \textbf{laboratories}: practical classes taught by professors, lecturers (rarely) or \acrshort{ta}s (\acrlong{ta}s), which require a computer or other external tools (e.g., lab equipment, electrical equipment, etcetera), which take part in dedicated lab rooms and in which only a group or subgroup of students participates
        \item \textbf{seminars}: practical classes taught by professors or lecturers, which only require a pen, paper, and a whiteboard, which take part in classrooms and in which only a group or subgroup of students participates
    \end{itemize}
    
    The primary forms of evaluation are theoretical exams, homework, projects, tests, practical exams, and research assignments.
        
\section{Proposed functionalities} \label{1:functionalities}

    Given the environment described in section \ref{1:environment}, we can propose a number of important actions that the student should be able to do within our application:
    
    \begin{itemize}
        \setlength{\topsep}{0.5pt}
        \setlength{\itemsep}{0.5pt}
        \setlength{\parsep}{0.5pt}
        \item \textbf{access, modify and filter a list of platforms provided by the university} (see section \ref{1:university_platforms}), as well as view information about their use
        \item collaboratively \textbf{create and edit a timetable} which includes class information as well as various events (based on the types described in section \ref{1:class_structure})
    \end{itemize}
    
    Additionally, the application can provide additional, useful features such as:
    
    \begin{itemize}
        \setlength{\topsep}{0.5pt}
        \setlength{\itemsep}{0.5pt}
        \setlength{\parsep}{0.5pt}
        \item \textbf{campus map and navigation information}
        \item \textbf{news, FAQs and other relevant information} to which students can contribute
        \item \textbf{authentication} (ideally linked to the common credentials used for the other platforms)
        \item \textbf{localization}: the application should provide a \acrshort{ui} (\acrlong{ui}) in both Romanian (for local students) and English (for international students)
    \end{itemize}

\section{Goals} \label{1:goals}

    We believe that the students understand their own needs better than the faculty staff ever could. The university lacks the agility to supply useful technologies aiding students in managing their resources and schedule. Therefore, we believe that an application built, maintained and managed by students has the best chance of providing a much-needed helping hand for students juggling with a vast amount of information coming from the university.
    
    This application should be intuitive and easy to use, as well as allow customization for any student. Above all, it should be particularly helpful for first-year students by offering an easy way to find out anything they want to know about the ins and outs of their new university life.
    
    Students can contribute to the application by submitting a \acrshort{pr} (\acrlong{pr}) into the public repository\footnote{https://github.com/acs-upb-mobile/acs-upb-mobile/}, which means that it can easily be kept up-to-date with their needs and wants.
    
    Ultimately, the application aims to become an integral part of students' lives and contribute to their education, all the while decreasing the feelings of anxiety caused by being bombarded with new, often disorganized information.

\section{Outline} \label{1:outline}
    \textbf{Chapter \ref{chapter2}} outlines the current state of the art, in terms of existing applications with a similar or related purpose to our application.
    
    \textbf{Chapter \ref{chapter3}} presents the methods used for our user study and focuses on what we learned from the survey we created.
    
    \textbf{Chapter \ref{chapter4}} portrays the iterative design process we went through in order to come up with the \acrshort{ux} and \acrshort{ui} of our application. It includes results from another step of our user study - focus groups and feedback requests.
    
    \textbf{Chapter \ref{chapter5}} describes our implementation - the technologies we used as well as the system architecture and automated testing/deployment pipelines.
    
    \textbf{Chapter \ref{chapter6}} provides an overview of the conclusions we've reached throughout the paper, including discussion about possible caveats and future improvements.