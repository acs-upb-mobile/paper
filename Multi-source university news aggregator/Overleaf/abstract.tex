\noindent

\textit{Această lucrare prezintă implementarea unui modul de agregator de știri, soluție ce își propune să ajute la centralizarea postărilor și noutăților din facultate. Modulul este integrat într-o aplicație deja existentă și are ca scop atât centralizarea informațiilor de pe mai multe platforme, într-un singur feed central, cât și încurajarea studenților de a contribui activ în construirea acestuia.}

\textit{Din cauza situației pandemice din ultimii ani, viața studenților s-a mutat integral în online, iar buna informarea a acestora a devenit dependentă de o multitudine de platforme online eterogene. Altfel spus, pentru ca un student să rămână constant la curent cu noutățile din facultate, acesta e nevoit să navigheze zilnic pe o serie de platforme și să filtreze o cantitate mare de conținut până să ajungă la informațiile relevante pentru el. Această lucrare tratează problema din perspectiva studenților de la Facultatea de Automatică și Calculatoare, Universitatea POLITEHNICA București.}

\textit{Încercăm să analizăm problema și impactul ei prin feedback-ul oferit de studenți și să înțelegem cum alte instituții universitare abordează acest subiect. Pe baza informațiilor colectate, dorim să înaintăm o soluție de agregator disponibilă pe o aplicație mobilă open-source contribuită de numeroși studenți în ultimii doi ani. }

\vspace*{2\baselineskip}

\textit{This paper presents the implementation of a content aggregator module that helps centralize the news related to university life. This module is integrated into an existing mobile application and is aimed at bringing the information from many platforms into one single centralized feed. Simultaneously, this feature should encourage the students to contribute to the feed and populate it with relevant news.}

\textit{As a direct consequence of the recent pandemic, the students' lives shifted into an online environment, and their information became dependent on several heterogeneous online platforms. In other words, for students to be aware of the official and extracurricular faculty news, they must navigate a series of platforms daily and filter a large quantity of information to reach the news they care about. This paper treats this problem from the students' perspective from the Faculty of Automatic Control and Computer Science, University POLITEHNICA of Bucharest.}

\textit{Based on the feedback provided by students, we try to analyze this problem and understand how other universities tackle this issue. Our proposed aggregator solution is built over an already available mobile application that many students contributed to in the last two years.}

\vspace*{\fill}